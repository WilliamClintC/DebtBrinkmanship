% Options for packages loaded elsewhere
\PassOptionsToPackage{unicode}{hyperref}
\PassOptionsToPackage{hyphens}{url}
\PassOptionsToPackage{dvipsnames,svgnames,x11names}{xcolor}
%
\documentclass[
  12pt]{article}

\usepackage{amsmath,amssymb}
\usepackage{iftex}
\ifPDFTeX
  \usepackage[T1]{fontenc}
  \usepackage[utf8]{inputenc}
  \usepackage{textcomp} % provide euro and other symbols
\else % if luatex or xetex
  \usepackage{unicode-math}
  \defaultfontfeatures{Scale=MatchLowercase}
  \defaultfontfeatures[\rmfamily]{Ligatures=TeX,Scale=1}
\fi
\usepackage{lmodern}
\ifPDFTeX\else  
    % xetex/luatex font selection
\fi
% Use upquote if available, for straight quotes in verbatim environments
\IfFileExists{upquote.sty}{\usepackage{upquote}}{}
\IfFileExists{microtype.sty}{% use microtype if available
  \usepackage[]{microtype}
  \UseMicrotypeSet[protrusion]{basicmath} % disable protrusion for tt fonts
}{}
\makeatletter
\@ifundefined{KOMAClassName}{% if non-KOMA class
  \IfFileExists{parskip.sty}{%
    \usepackage{parskip}
  }{% else
    \setlength{\parindent}{0pt}
    \setlength{\parskip}{6pt plus 2pt minus 1pt}}
}{% if KOMA class
  \KOMAoptions{parskip=half}}
\makeatother
\usepackage{xcolor}
\setlength{\emergencystretch}{3em} % prevent overfull lines
\setcounter{secnumdepth}{5}
% Make \paragraph and \subparagraph free-standing
\ifx\paragraph\undefined\else
  \let\oldparagraph\paragraph
  \renewcommand{\paragraph}[1]{\oldparagraph{#1}\mbox{}}
\fi
\ifx\subparagraph\undefined\else
  \let\oldsubparagraph\subparagraph
  \renewcommand{\subparagraph}[1]{\oldsubparagraph{#1}\mbox{}}
\fi


\providecommand{\tightlist}{%
  \setlength{\itemsep}{0pt}\setlength{\parskip}{0pt}}\usepackage{longtable,booktabs,array}
\usepackage{calc} % for calculating minipage widths
% Correct order of tables after \paragraph or \subparagraph
\usepackage{etoolbox}
\makeatletter
\patchcmd\longtable{\par}{\if@noskipsec\mbox{}\fi\par}{}{}
\makeatother
% Allow footnotes in longtable head/foot
\IfFileExists{footnotehyper.sty}{\usepackage{footnotehyper}}{\usepackage{footnote}}
\makesavenoteenv{longtable}
\usepackage{graphicx}
\makeatletter
\def\maxwidth{\ifdim\Gin@nat@width>\linewidth\linewidth\else\Gin@nat@width\fi}
\def\maxheight{\ifdim\Gin@nat@height>\textheight\textheight\else\Gin@nat@height\fi}
\makeatother
% Scale images if necessary, so that they will not overflow the page
% margins by default, and it is still possible to overwrite the defaults
% using explicit options in \includegraphics[width, height, ...]{}
\setkeys{Gin}{width=\maxwidth,height=\maxheight,keepaspectratio}
% Set default figure placement to htbp
\makeatletter
\def\fps@figure{htbp}
\makeatother

\addtolength{\oddsidemargin}{-.5in}%
\addtolength{\evensidemargin}{-1in}%
\addtolength{\textwidth}{1in}%
\addtolength{\textheight}{1.7in}%
\addtolength{\topmargin}{-1in}%
\makeatletter
\makeatother
\makeatletter
\makeatother
\makeatletter
\@ifpackageloaded{caption}{}{\usepackage{caption}}
\AtBeginDocument{%
\ifdefined\contentsname
  \renewcommand*\contentsname{Table of contents}
\else
  \newcommand\contentsname{Table of contents}
\fi
\ifdefined\listfigurename
  \renewcommand*\listfigurename{List of Figures}
\else
  \newcommand\listfigurename{List of Figures}
\fi
\ifdefined\listtablename
  \renewcommand*\listtablename{List of Tables}
\else
  \newcommand\listtablename{List of Tables}
\fi
\ifdefined\figurename
  \renewcommand*\figurename{Figure}
\else
  \newcommand\figurename{Figure}
\fi
\ifdefined\tablename
  \renewcommand*\tablename{Table}
\else
  \newcommand\tablename{Table}
\fi
}
\@ifpackageloaded{float}{}{\usepackage{float}}
\floatstyle{ruled}
\@ifundefined{c@chapter}{\newfloat{codelisting}{h}{lop}}{\newfloat{codelisting}{h}{lop}[chapter]}
\floatname{codelisting}{Listing}
\newcommand*\listoflistings{\listof{codelisting}{List of Listings}}
\makeatother
\makeatletter
\@ifpackageloaded{caption}{}{\usepackage{caption}}
\@ifpackageloaded{subcaption}{}{\usepackage{subcaption}}
\makeatother
\makeatletter
\@ifpackageloaded{tcolorbox}{}{\usepackage[skins,breakable]{tcolorbox}}
\makeatother
\makeatletter
\@ifundefined{shadecolor}{\definecolor{shadecolor}{rgb}{.97, .97, .97}}
\makeatother
\makeatletter
\makeatother
\makeatletter
\makeatother
\ifLuaTeX
  \usepackage{selnolig}  % disable illegal ligatures
\fi
\usepackage[]{natbib}
\bibliographystyle{agsm}
\IfFileExists{bookmark.sty}{\usepackage{bookmark}}{\usepackage{hyperref}}
\IfFileExists{xurl.sty}{\usepackage{xurl}}{} % add URL line breaks if available
\urlstyle{same} % disable monospaced font for URLs
\hypersetup{
  pdftitle={Is US Debt Brinkmanship a Debt Crisis Without Default?},
  pdfauthor={William Clinton Co},
  pdfkeywords={safe asset shortage, increasing public debt},
  colorlinks=true,
  linkcolor={blue},
  filecolor={Maroon},
  citecolor={Blue},
  urlcolor={Blue},
  pdfcreator={LaTeX via pandoc}}


\begin{document}


\def\spacingset#1{\renewcommand{\baselinestretch}%
{#1}\small\normalsize} \spacingset{1}


%%%%%%%%%%%%%%%%%%%%%%%%%%%%%%%%%%%%%%%%%%%%%%%%%%%%%%%%%%%%%%%%%%%%%%%%%%%%%%

\date{August 25, 2023}
\title{\bf Is US Debt Brinkmanship a Debt Crisis Without Default?}
\author{
William Clinton Co\thanks{We would like to express my gratitude to
Jonathan Graves for helpful feedback ,guidance and the opportunity to
participate.}\\
Department of Economics, The University of British Columbia\\
}
\maketitle

\bigskip
\bigskip
\begin{abstract}
Under the backdrop of increasing public debt, ``debt crisis without
default'' and safe asset shortages, we investigate how US debt
brinkmanship plays a role into mentioned topics.
\end{abstract}

\noindent%
{\it Keywords:} safe asset shortage, increasing public debt
\vfill

\newpage
\spacingset{1.9} % DON'T change the spacing!
\ifdefined\Shaded\renewenvironment{Shaded}{\begin{tcolorbox}[borderline west={3pt}{0pt}{shadecolor}, breakable, enhanced, interior hidden, boxrule=0pt, frame hidden, sharp corners]}{\end{tcolorbox}}\fi

\hypertarget{sec-l}{%
\section{Literature}\label{sec-l}}

\hypertarget{increasing-public-debt}{%
\subsection{Increasing Public Debt}\label{increasing-public-debt}}

Constantly increasing public debt has been a recent development
throughout recent history \citep{mitchener2023}. This raises the
question of how will governments deal with rising debt burdens going
forward. As debt increases, cost of borrowing increases. Will
governments internalize the increase of cost of borrowing?

\hypertarget{debt-crisis-without-default}{%
\subsection{Debt Crisis Without
Default}\label{debt-crisis-without-default}}

It has also been noted we have debt crisis without default has become
more common, wherein there was a near missed payment but never a
default, as exemplified in Greece Portugal and Spain during 2010-2012
\citep{mitchener2023}. To take this into account, some have proposed to
redefine a debt crisis as yield spreads of 1000 basis points, also known
as spread spikes \citep{broner2013, aguiar, krishnamurthy}.

What is interesting is that ``debt crisis without default'' function as
if a default had actually occurred. Output declines associated with a
default occur even if a default had not occurred. The anticipation of a
potential default was sufficient for output declines.
\citep{yeyati2011}.

The literature analyzes possible channels of said output decline. One
channel is higher yields. There are different justifications for this
channel namely increased external financing cost to importers
\citep{mendoza2012}, decrease in external domestic firm
borrowing\citep{corsetti2012, das2010, gourinchas2016} or tightening of
credit against loses on bank's balance
sheets\citep{arellano, ferrando2017}. Another channel is credit rating
downgrades, which reduces leverage and investments \citep{almeida2017}.
These channels were robust even when considering high frequency CDS risk
premium data and SME (small and medium sized enterprise) surveys
\citep{brutti2015, bahaj2020, almeida2017}.

Despite the fact that US debt brinkmanship raises yields
\citep{nippani2017}, the current literature has yet to consider if US
debt ceiling brinkmanship is akin to a debt crisis without default,
wherein yield spikes cause US output to decline in anticipation of
default.

\hypertarget{safe-asset-shortage}{%
\subsection{Safe Asset Shortage}\label{safe-asset-shortage}}

Another pertinent question is the many creditors willing to lend to
highly indebted sovereigns. Currently we are in a safe asset shortage,
such that we are coming closer to the effective lower bound, wherein
central banks could not decrease interest rates any further as needed.
This shortage is a key source of fragility in the economy, dubbed the
``safety trap'' \citep{caballero2017} . Similarly, the current
literature has yet to consider if US debt ceiling brinkmanship
contributes to this phenomenon.

Rising US Government Debt

Debt to GDP ratios have been a notable trend

\hypertarget{advanced-economy}{%
\subsection{Advanced Economy}\label{advanced-economy}}

US debt to GDP ratios are approaching levels seen in World War II
\citep{yared2019} . With projections expecting further increases
\citep{congressionalbudgetoffice2023}.

Talk about the reasons like old age.

\hypertarget{introduction}{%
\section{Introduction}\label{introduction}}

The US treasury yield occupies the status as the biggest and most liquid
market, wherein its yield is a significant determinant of yields
globally. This phenomenon would be described as the ``global factor''
becoming increasingly more important determinant of yields ,against
specific ``country'' factors \citep{mauro2002}. Thus, studying the
properties of US' yields would be important.
\citep{rozada2006, gonzález-rozada2008, longstaff2011}. We shall study
US' yields in the context of debt ceiling brinkmanship. Furthermore,
current literature on debt focuses on events like Greece or Argentina,
less work has been done with consideration to US debt ceiling.

\hypertarget{public-debt-and-debt-brinkmanship}{%
\subsection{Public Debt and Debt
Brinkmanship}\label{public-debt-and-debt-brinkmanship}}

Previous literature establishes the recent development of increasing
high public debt \citep[ ]{mitchener2023}. While others note that debt
brinkmanship has become more and more worse \citep{berman} , evident by
the increasing trend of passing debt limit suspension vs raises.
Insiders and analyst mention how normalized brinkmanship has become
\citep{bivens}. The causes of this are many. Some mention how the rise
of populism made the US government less responsive to business leaders
\citep{cook2023} . The incentive structures in place have shifted to
financial contributions being raised online in small amounts rather than
big donations from prominent interest. Gerrymandering has also been a
variable. With the rise of ``custom-designed'' districts, swing
districts are becoming rare. This shifts the incentive structure to
cater less to independents or moderates, which leads to less
compromises. The number of moderates currently in congress is
significantly less compared to the 1970s \citep{desilver2022},

does brinkmanship intensity improve debt to gdp or just debt is it a
sign

to be specific us govenrment debt to gdp ratios are at an all time
historical high site sources

overalld ebt is high as well

there are numerous reasons for tis old people

talk about atx moothing theories and their failure maybe brinkmanship

possible causes????

talk about the sfa asset privison theory suppose that we haev debt
brinkmanship then the theory suggest we would increase public debt
becuase this would mean less safe assets?

or would it mean that less debt because now we have credibility
uncertainty risk

does brinkmanship constrain fiscal debt

TALK ABOUT INequality political gridlock talk abotu data and the
distribution of different opinions

We investigate the link between the two.

We plot US debt limit increases along with world global change in
debt/GDP ratios. We also plot the frequency of debt raises/suspensions
to identify trends. We will also take note of rating agency negative
outlooks from the top 3 rating agencies. Taking inspiration from

\includegraphics[width=4.8125in,height=\textheight]{style-guide/Debt plot.png}

\hypertarget{how-will-governments-react-to-the-increasing-cost-of-borrowing}{%
\subsection{How will governments react to the increasing cost of
borrowing?}\label{how-will-governments-react-to-the-increasing-cost-of-borrowing}}

We construct a data set of X-dates, dates where the US government will
supposedly run out of money. This is done by analyzing the maximum ex
ante yield curves and CDS prices. An example would be

\includegraphics[width=4.625in,height=\textheight]{style-guide/x-date-estmation.jpg}

{[}\citet{boesler}{]}\citep{steinmetz-silber} .

In here we see the peak of the yield curve would correspond to the
``x-date''. Similarly,

\includegraphics[width=5.09375in,height=\textheight]{style-guide/CDS-x-date.png}

{[}\citet{rao2023}{]}\citep{benzoni} .

In here we see the peaks of CDS prices correspond to the ``x-date''.

We then analyze changes in CDS prices and yields, using data from
Bloomberg. We use official whitehouse data to get dates of debt limit
increase. We build on prior work which uses the 1000 basis points as a
benchmark. We isolate brinkmanship with a 1000 basis point increase
against those without. An example would be

\begin{longtable}[]{@{}
  >{\raggedright\arraybackslash}p{(\columnwidth - 10\tabcolsep) * \real{0.1644}}
  >{\raggedright\arraybackslash}p{(\columnwidth - 10\tabcolsep) * \real{0.1644}}
  >{\raggedright\arraybackslash}p{(\columnwidth - 10\tabcolsep) * \real{0.1781}}
  >{\raggedright\arraybackslash}p{(\columnwidth - 10\tabcolsep) * \real{0.1644}}
  >{\raggedright\arraybackslash}p{(\columnwidth - 10\tabcolsep) * \real{0.1644}}
  >{\raggedright\arraybackslash}p{(\columnwidth - 10\tabcolsep) * \real{0.1644}}@{}}
\toprule\noalign{}
\begin{minipage}[b]{\linewidth}\raggedright
X-date
\end{minipage} & \begin{minipage}[b]{\linewidth}\raggedright
Date of Increase
\end{minipage} & \begin{minipage}[b]{\linewidth}\raggedright
Negotiation length=(X-date)-Date of Increase
\end{minipage} & \begin{minipage}[b]{\linewidth}\raggedright
CDS
\end{minipage} & \begin{minipage}[b]{\linewidth}\raggedright
CDS1000(1000 basis points or more)
\end{minipage} & \begin{minipage}[b]{\linewidth}\raggedright
Yields
\end{minipage} \\
\midrule\noalign{}
\endhead
\bottomrule\noalign{}
\endlastfoot
\(x_1\) & \(d_1\) & \(n_{1,yes}=x_1-d_1\) & \(c_1\) & yes & \(y_1\) \\
\(x_2\) & \(d_2\) & \(n_{1,no}=x_2-d_2\) & \(c_2\) & no & \(y_2\) \\
\ldots. & \ldots.. & \ldots. & \ldots.. & \ldots.. & \ldots{} \\
\end{longtable}

We investigate if debt ceiling negotiations settle faster given a sharp
increase in cost. We compute \(\bar{n_{y}}\) , average negotiation
length with a spread spike and compare this to \(\bar{n_{n}}\),
negotiation length with no spike.

We also run regression
\(NegoLength=\beta_1\Delta CDS+\beta_2\Delta Yields+\beta_3D_{neg-outlook}\),
such that we investigate weather debt ceiling negotiations will settle
earlier given a bigger increased in cost of capital. We split cost of
capital into three components CDS prices, yields and rating agency
downgrades.

We investigate trends overtime by \textbf{plotting} negation length on
the y axis against date of increase on the x axis.

We study how brinkmanship affects country yield spreads. we take
inspiration from the data set by \citep{meyer2022} as it relates data on
debt ceiling brinkmanship \citep{reinhart2008}.

\includegraphics[width=3.94792in,height=\textheight]{style-guide/overtime_brink_2.png}

\hypertarget{contributor-to-safe-asset-shortage}{%
\subsection{Contributor to safe asset
shortage?}\label{contributor-to-safe-asset-shortage}}

Ever since the 2008 financial crisis risk premiums have not returned to
prior levels\citep{caballero2017}.We investigate if debt brinkmanship
contributes to this phenomenon as well. If so then there would be an
argument to abolish the system on a global welfare standpoint. We take
inspiration from

\includegraphics[width=6.10417in,height=\textheight]{style-guide/1_year_ERP.png}

We construct a similar graph as above taking debt ceiling dates.

Using time t, as the time of debt increase. We graph a line representing
the average change in 1 year expected risk premium. Another line
represents the average change in 1 year treasury yields. We construct
the graph below with mentioned variables \citep{duarte2015}.

\includegraphics[width=5.21875in,height=\textheight]{style-guide/1_year_ERP_parallel_trends.jpeg}

\hypertarget{debt-crisis-without-default-1}{%
\subsection{Debt Crisis without
Default?}\label{debt-crisis-without-default-1}}

We then investigate if debt ceiling brinkmanship can be characterized as
a debt crisis without default, as defined by prior literature. We graph

\includegraphics[width=4.875in,height=\textheight]{style-guide/1_year_ERP_parallel_trends.jpeg}

the y axis would represent GDP. We then make 3 lines corresponding to
the following attributes advanced countries, developing countries and
China. We will use IMF definition and ifs data-set to accomplish this.
By doing this, we investigate if the output decline \citep{yeyati2011}
is present.

We consider China for the following reasons. China's rise to the world
stage has been marked with capital exports that significantly alter
global yields \citep{alfaro2014, gourinchas}. In fact, China's lending
portfolio surpasses that of the World Bank \citep{horn2021} . We use
horns data-set. to analyze this.

We also propose a similar graph using imports on the y axis in line with
\citep{mendoza2012}. Similarly, We investigate debt/market cap levels by
firms \citep{corsetti2012, das2010, gourinchas2016} and investments
\citep{almeida2017}.

Lastly, there is work that shows US debt brinkmanship directly increase
treasury yields and borrowing cost \citep{nippani2017}. We investigate
and compare the increased cost in borrowing and treasury yield of the US
against countries the defaulted or are under the category of ``debt
crisis without default''. Examples of such countries would be Argentina,
Spain, Venezuela, etc. The study would be implemented in a parallel
trends assumptions test as shown in the figure above.

\hypertarget{future-considerations}{%
\section{Future Considerations}\label{future-considerations}}

Future considerations are as follows. Rubenstein bargaining model.
Trembling hand model. Congressmen attendance. Measures of reserve
currency. talk about nippani 2017 resignation


\renewcommand\refname{References}
  \bibliography{bibliography.bib}


\end{document}
